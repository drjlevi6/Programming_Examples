\font\fourteenbold=cmb10 scaled \magstep2
\font\eightrm=cmr8
\font\ninerm=cmr9
\font\tenmib=cmmib10
\centerline {\fourteenbold{The Newton-Raphson(?) Method}}
\centerline{\fourteenbold{For Computing Square Roots}}\vskip 12pt
\footline{\eightrm Newton-Raphson.dvi\hfil\ninerm\the\pageno
	\hfil$\qquad$\eightrm J. Levi, May 2001}
\parindent 0pt
\parskip 12pt

The following is an attempt to define and validate what I believe is called the Newton-Raphson approximation method for extracting the positive square root of a positive number. (For the rest of this paper, ``square root'' will be taken to mean a number's positive square root.)

\proclaim Definition. Let $a$ be a positive real number. A \hbox{\bf Newton-Raphson sequence on {\tenmib a}} is a sequence of real numbers $\{x _0,\ x_1,\dots\}$ such that
$$\eqalign{
	&(1)\qquad x_0\hbox{ is positive;}\cr
	&(2)\qquad x_{n+1}={\left(x_n +{a\over x_n}\right)/2,}\qquad n=0,1,\dots
}$$

\proclaim Theorem. Let $a$ be positive, $\{x_n\}$ a Newton-Raphson sequence on a. Then $\lim_{n\to\infty}x_n$ exists and
$$\lim_{n\to\infty}x_n = \sqrt a.$$

\proclaim Lemma. Let $a$ be positive, $\{x_n\}$ a Newton-Raphson sequence on a. Unless $x_0 = \sqrt a,$ then for all positive integers n, $x_n > \sqrt a.$

{\bf Proof of Lemma.} We consider the function
	$$\qquad\qquad y(x) = {\left(x+ {a\over x}\right)/2, } \qquad x>0.$$
Then $y$ is differentiable and	$$y'(x) = {\left(1 - {a\over{x^2}}\right)/2}.$$
Moreover, $y'$ is differentiable and $$y''(x) = {a\over{x^3}}.$$

We see that $y'(x) = 0$ iff $x = \sqrt a;$ since $y''(\sqrt a)$ is positive, $y$ is minimal at $\sqrt a.$ Since $y(\sqrt a) = \sqrt a,\>$ for all (positive) $x\ne\sqrt a, y(x) > \sqrt a.$ Then $y(y(x))$ is defined and $y(y(x))> \sqrt a;\>$ similarly for $y(y(y(x))),$ and so on. But $x_1=y(x_0),\ x_2=y(x_1)=y(y(x_0)),\dots$ Hence $x_0\neq \sqrt a\Rightarrow x_1, x_2, x_3\dots > \sqrt a.$

{\bf Proof of Theorem.} If $x_0=\sqrt a$ the proof is trivial, since all of the $x_n$ equal $\sqrt a.$ Otherwise, we must show that $|x_n-\sqrt a| \rightarrow 0.$ We note that for $i=0,1,\dots$
	$$\eqalign{\qquad{|x_{i+1}-\sqrt a|}
		&={\left|{\left(x_i + {a\over x_i}\right)}/2-\sqrt a\right|}\cr
		&={{|{x_i}^2 -2x_i \sqrt a+ a|}\over{2x_i}}\cr
		&={{{(x_i-\sqrt a)}^2}\over{2x_i}};}$$
therefore	$${\left|{{x_{i+1}-\sqrt a}\over{x_i-\sqrt a}}\right|}
				={{|x_i-\sqrt a|}\over{2x_i}}.$$
From the Lemma we know that $x_i-\sqrt a>0$ as long as $i>0,$ so
	$$\eqalign{{\left|{{x_{i+1}-\sqrt a}\over{x_i-\sqrt a}}\right|}&=
		{{x_i-\sqrt a}\over{2x_i}}\cr &<{x_i\over{2x_i}}={1/2}.}$$
Hence for any integer $m>0,$
	$$\eqalign{{\left|{{x_{i+m}-\sqrt a}\over{x_i-\sqrt a}}\right|}&=
		\prod_{j=1}^m{\left|{x_{i+j}-\sqrt a}\over{x_{i+j-1}-\sqrt a}\right|}\cr
		&<\prod_{j=1}^m{\left(1\over2\right)}\cr
		&= {\left(1\over2\right)}^m.}$$
Setting $i=1$ gives
	$${|x_{1+m}-\sqrt a|}<{{|x_1-\sqrt a|}\over{2^m}};$$
Finally, let $n=m+1$ (and $n>1);$ then
	$${|x_n-\sqrt a|}<{{|x_1-\sqrt a|}\over{2^{n-1}}}$$
and the theorem is proved.\bye